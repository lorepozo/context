\documentclass[12pt,letterpaper]{article}

\renewcommand{\baselinestretch}{1.2}

\pagenumbering{gobble}


\title{{\large SuperUROP 2016-2017 Proposal:}\\
  {\Large Learning with Semi-Structured Modular Concepts on a Symbolic Tagged
Representation}}
\author{Lucas E. Morales \texttt{\{lucasem\}}}
\date{}


\begin{document}

\maketitle

Theories, when learned, are constructed from combinations of concepts
already known. I aim to organize a system for modular concept discovery
influenced by previous works. Concepts may establish structured relations
between themselves, such as real-valued functions which depend on the real
numbers, which in turn depend on a construction of the real number system,
which depends on propositional logic, etc. Conceptual relationships may not
necessarily be hierarchical: concepts may share an interface and are
hot-swappable, such as with the choice of a bag versus a box when confronted
with a problem of containment. By maintaining these relationships, learning
may be achieved with varying levels of granularity and ambiguity in the
concepts involved. Relationships between concepts are stored in symbolic
tags, a component of their representation, to permit both primitive and
higher-order connections that may depend on other concepts.

I intend to formalize this representation of knowledge, expand on existing
learning systems to incorporate this feature, and analyze results of such
modified approaches. I may determine an applicable problem domain to design
a different learning system more tailored to the idea of semi-structured
modular concepts.

% AMB to reorganize parameterized inputs until proposition matches (SAT
% Solver)

% Heirarchy allows for metareasoning on higher levels of abstraction
%  [2] construct a metatheory/schema for the solution, then fill in the gap
% [2] Kevin Ellis, Owen Lewis. Metareasoning in Symbolic Domains. NIPS 2015 Workshop on Bounded Optimality and Rational Metareasoning

% Useful links:
%  - http://edechter.github.io/publications/DBLP_conf_ijcai_DechterMAT13.pdf
%  - http://www.cs.berkeley.edu/~jordan/papers/liang-jordan-klein-icml10.pdf
%  - http://edechter.github.io/publications/dechter2015latent.pdf
%  - http://www.mit.edu/~ellisk/symdimred.pdf
%  - http://www.mit.edu/~ellisk/metareasoning.pdf

%\begin{thebibliography}{}
%
%\bibitem{dechter13}
%  Dechter, E., Malmaud, J., Adams, R. P., & Tenenbaum, J. B. (2013).
%  Bootstrap Learning via Modular Concept Discovery.
%  \emph{Proceedings of the 23rd International Joint Conference on%
%    Artificial Intelligence}
%
%\bibitem{liang10}
%  Liang, P., Jordan, M. I., \& Klein, D. (2010).
%  Learning programs: a hierarchical bayesian approach
%  \emph{International Conference on Machine Learning}: 639–646
%
%\bibitem{church}
%  Church, A. (1941)
%  \emph{The Calculi of Lambda-Conversion}
%  Princeton University Press
%
%\bibitem{schonfinkel}
%  Sch\"{o}nfinkel, M. (1924).
%  \"{U}ber die bausteine der mathematischen logik.
%  \emph{Mathematische Annalen} (in German) 92: 305—316
%
%\bibitem{curry}
%  Curry, H. B. (1930)
%  Grundlagen der kombinatorischen logik
%  \emph{American Journal of Mathematics} (in German) 52 (3): 509–536.
%
%\end{thebibliography}
\end{document}

